\documentclass[main.tex]{subfiles}
\begin{document}
    \subsection{Рассматриваемые определения}
\begin{itemize}
		\item Нормальное распределение
		    \begin{equation}
		    	N(x, 0, 1) = \frac{1}{\sqrt{2\pi}}e^{\frac{-x^2}{2}} \label{norm} 
			\end{equation}
		\item Распределение Коши
		    \begin{equation}
				C(x, 0, 1) = \frac{1}{\pi}\frac{1}{x^2+1} \label{koshi}
			 \end{equation} 
		\item Распределение Пуассона
	        \begin{equation}
				P(k, 10) = \frac{10^k}{k!}e^{-10}\label{puasson}
			\end{equation}
		\item Равномерное распределение 
		    \begin{equation}
				U(x, -\sqrt{3}, \sqrt{3}) =	\begin{cases}\frac{1}{2\sqrt{3}} 
				&\text{$при |x|\leq \sqrt{3}$}\\0 &\text{$при |x|>\sqrt{3}$}\end{cases}\label{uni} 
			\end{equation}
	\end{itemize}
	
	\subsection{Гистограмма}
	\subsubsection{Определение}
	\noindent \textit{Гистограмма} в математической статистике — это функция, приближающая плотность вероятности некоторого распределения, построенная на основе выборки из него.
	
	\subsubsection{Графическое описание}
	\noindent Графически гистограмма строится следующим образом. Сначала множество значений, которое может принимать элемент выборки, разбивается на несколько интервалов. Чаще всего эти интервалы берут одинаковыми, но это не является строгим требованием. Эти интервалы откладываются на горизонтальной оси, затем над каждым рисуется прямоугольник. Если все интервалы были одинаковыми, то высота каждого прямоугольника пропорциональна числу элементов выборки, попадающих в соответствующий интервал. Если интервалы разные, то высота прямоугольника выбирается таким образом, чтобы его площадь была пропорциональна числу элементов выборки, которые попали в этот интервал.
	
	\subsubsection{Использование}
	\noindent Гистограммы применяются в основном для визуализации данных на начальном этапе статистической обработки. \newline Построение гистограмм используется для получения эмпирической оценки плотности распределения случайной величины. Для построения гистограммы наблюдаемый диапазон изменения случайной величины разбивается на несколько интервалов и подсчитывается доля от всех измерений, попавшая в каждый из интервалов. Величина каждой доли, отнесенная к величине интервала, принимается в качестве оценки значения плотности распределения на соответствующем интервале.
	
    \subsection{Вариационный ряд}
	\noindent Вариационным рядом называется последовательность элементов выборки, расположенных в неубывающем порядке. Одинаковые элементы повторяются.
	Запись вариационного ряда: $x_{(1)}, x_{(2)}, \ldots, x_{(n)}$.
	Элементы вариационного ряда $x_{(i)} (i = 1, 2, \ldots, n)$ называются порядковыми статистиками.
	
	\subsection{Выборочные числовые характеристики}
	\noindent С помощью выборки образуются её числовые характеристики. Это числовые характеристики дискретной случайной величины $X^{*}$, принимающей выборочные значения $x_{(1)}, x_{(2)}, \ldots, x_{(n)}$.
	
	\subsubsection{Характеристики положения}
	\begin{itemize}
		\item Выборочное среднее 
		\begin{equation}
			\overline{x} = \frac{1}{n}\sum_{i=1}^{n}{x_i}
		\end{equation}
		\item Выборочная медиана 
		\begin{equation}
			med x = 
			\begin{cases}
			    x_{(l+1)} &\text{$ n=2l+1$}\\
				\frac{x_{(l)} + x_{(l+1)}}{2} &\text{$ n=2l$}
			\end{cases}
		\end{equation}
		\item Полусумма экстремальных выборочных элементов 
		\begin{equation}
			 z_R = \frac{x_{(1)} + x_{(n)}}{2}
		\end{equation}
		\item Полусумма квартилей 
		\newline Выборочная квартиль $z_p$ порядка $p$ определяется формулой \begin{equation}
			z_p = 
			\begin{cases}
		       x_{([np]+1)} &\text{$np - $дробное}\\
		       x_{(np)}&\text{$np - $целое}
		    \end{cases}
		\end{equation}
		Полусумма квартилей \begin{equation}
			z_Q = \frac{z_{1/4} + z_{3/4}}{2}
		\end{equation}
		\item Усечённое среднее
		\begin{equation}
			z_{tr} = \frac{1}{n-2r}\sum_{i=r+1}^{n-r}{x_{(i)}}, r\approx\frac{n}{4}	   	
		\end{equation}
	\end{itemize}

	\subsubsection{Характеристики рассеяния}
	Выборочная дисперсия
	\begin{equation}
		D = \frac{1}{n}\sum_{i=1}^{n}{(x_i-\overline{x})^2}
	\end{equation}
	
	\subsection{Боксплот Тьюки}
	
	\subsubsection{Определение}
	\noindent \textit{Боксплот} (англ. box plot) — график, использующийся в описательной статистике, компактно изображающий одномерное распределение вероятностей.

	\subsubsection{Описание}
    \noindent Такой вид диаграммы в удобной форме показывает медиану, нижний и верхний квартили и выбросы. Несколько таких ящиков можно нарисовать бок о бок, чтобы визуально сравнивать одно распределение с другим; их можно располагать как горизонтально, так и вертикально. Расстояния между различными частями ящика позволяют определить степень разброса (дисперсии) и асимметрии данных и выявить выбросы.

    \subsubsection{Построение}
    \noindent Границами ящика служат первый и третий квартили, линия в середине ящика — медиана. Концы усов — края статистически значимой выборки (без выбросов). Длину «усов» определяют разность     первого квартиля и полутора межквартильных расстояний и сумма третьего квартиля и полутора межквартильных расстояний. Формула имеет вид
    \begin{equation}\label{boxplot:mustache}
    	{X_1 = Q_1-} \frac{3}{2}{(Q_3 - Q_1)},   {X_2 = Q_3+} \frac{3}{2}{(Q_3 - Q_1)}
    \end{equation}
        где $X_1$ — нижняя граница уса, $X_2$ — верхняя граница уса, $Q_1$ — первый квартиль, $Q_3$ — третий квартиль. Данные, выходящие за границы усов (выбросы), отображаются на графике в виде маленьких кружков.


    \subsection{Теоретическая вероятность выбросов}
    \noindent Встроенными средствами языка программирования Python в среде разработки PyCharm можно вычислить теоретические первый и третий квартили распределений ($Q_1^T$ и $Q_3^T$ соответственно). По формуле \eqref{boxplot:mustache} можно вычислить теоретические нижнюю и верхнюю границы уса ($X_1^T$ и $X_2^T$ соответственно). Выбросами считаются величины x: 
    \begin{equation} \label{boxplot:emisssions}
    	\left[
    	\begin{gathered}
    		x < X_1^T \\
	    	x > X_2^T \\
    	\end{gathered}
    	\right.
    \end{equation}
        Теоретическая вероятность выбросов 
    \begin{itemize}
	    \item для непрерывных распределений
	\begin{equation} \label{boxplot:emisProbContin}
		P_B^T = P(x<X_1^T) + P(x>X_2^T)=F(X_1^T) + (1-F(X_2^T))
	\end{equation}
	    \item для дискретных распределений
    	\begin{equation}\label{boxplot:emisProbDiscr}
	    	P_B^T = P(x<X_1^T)+P(x>x_2^T)=(F(X_1^T)-P(x=X_1^T))+(1-F(X_2^T))
    \end{equation}
    \end{itemize}
        где $F(X) = P(x\leq{X})$ - функция распределения

    \subsection{Эмпирическая функция распределения}
    \subsubsection{Статистический ряд}
    \noindent Статистическим рядом назовем совокупность, состоящую из последовательности $\displaystyle\{z_i\}_{i=1}^k$ попарно различных элементов выборки, расположенных по возрастанию, и последовательности $\displaystyle\{n_i\}_{i=1}^k$ частот, с которыми эти элементы содержатся в выборке.
    \subsubsection{Эмпирическая функция распределения}
    \noindent Эмпирическая функция распределения (э. ф. р.) - относительная частота события $X < x$, полученная по данной выборке:
    \begin{equation} \label{empiricalFunc}
    	F_n^*(x)=P^*(X<x).
    \end{equation}
    \subsubsection{Нахождение э. ф. р.}
    \noindent Для получения относительной частоты $P^*(X < x)$ просуммируем в статистическом ряде построенном по данной выборке все частоты $n_i$, для которых элементы $z_i$ статистического ряда меньше $x$. Тогда $P^*(X < x) = \frac{1}{n}\sum_{z_i<x}n_i$. Получаем   

    \begin{equation} \label{empiricalFunc:EFD}
    	F^*(x)=\frac{1}{n}\sum_{z_i<x}n_i.
    \end{equation}
        $F^*(x)-$ функция распределения дискретной случайной величины $X^*$, заданной таблицей распределения
    \begin{table}[H]
	    \centering
	\begin{tabular}{|c|c|c|c|c|}
	    	\hline
	    	$X^*$&$z_1$&$z_2$&...&$z_k$\\
	    	\hline
	    	$P$&$n_1/n$&$n_2/n$&...&$n_k/n$\\
	    	\hline
	    \end{tabular}
    	\caption{Таблица распределения}
    	\label{tab:my_label}
    \end{table}
    
    \noindent Эмпирическая функция распределения является оценкой, т. е. приближённым значением, генеральной функции распределения
    \begin{equation} \label{empiricalFunc:approx}
    	F_n^*(x)\approx F_X(x).
    \end{equation}


    \subsection{Оценки плотности вероятности}
    \subsubsection{Определение}
    \noindent Оценкой плотности вероятности $f(x)$ называется функция $\widehat{f}(x)$, построенная на основе выборки, приближённо равная $f(x)$
    \begin{equation} \label{densityEstim}
    	\widehat{f}(x)\approx f(x).
    \end{equation}

    \subsubsection{Ядерные оценки}
    \noindent Представим оценку в виде суммы с числом слагаемых, равным объёму выборки:
    \begin{equation} \label{KDE}
    	\widehat{f}_n(x)=\frac{1}{n h_n}\sum_{i=1}^n K\left(\frac{x-x_i}{h_n}\right).
    \end{equation}
        $K(u)$ - ядро, т. е. непрерывная функция, являющаяся плотностью вероятности, $x_1,...,x_n$ $-$ элементы выборки, а $\{h_n\}_{n\in\mathbb{N}}$ - последовательность элементов из $\mathbb{R}_+$ такая, что
    \begin{equation} \label{KDE:Prop}
    	h_n\xrightarrow[n\to\infty]{}0;\;\;\;n h_n\xrightarrow[n\to\infty]{}\infty.
    \end{equation}
        Такие оценки называются непрерывными ядерными.\\\\
        Гауссово ядро:
    \begin{equation} \label{KDE:Gauss}
    	K(u)=\frac{1}{\sqrt{2\pi}}e^{-\frac{u^2}{2}}.
    \end{equation}
        Правило Сильвермана:
    \begin{equation} \label{KDE:Silverman}
    	h_n=\left(\frac{4\hat{\sigma}^5}{3n}\right)^{1/5}\approx1.06\hat{\sigma}n^{-1/5},
    \end{equation}
        где $\hat{\sigma}$ - выборочное стандартное отклонение.
\end{document}